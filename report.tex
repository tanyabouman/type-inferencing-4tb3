\documentclass[12pt]{article}
\usepackage{cite}
\usepackage{url}
\usepackage{hyperref} % for inserting the link
\usepackage{calculation}
\usepackage{amsmath}
\usepackage{amssymb}
\usepackage{color}
\usepackage{enumitem} % for alpha enumerations
\usepackage{listings} % for putting code snippets
\lstset{
  breaklines=true % line breaks for listings
}
\usepackage{cancel} % for crossing out cancellations in math



\title{4TB3 } 
\author{Emily Ashworth, Tanya Bouman, Tonye Fiberesima \\ 
001402976, 001416669, 001231043 \\ 
ashworel, boumante, fiberet}
\date{April 9, 2018}

\begin{document}

\maketitle

Many advanced languages, such as Swift\cite{swifttypedocs} and Haskell\cite{haskelltypedocs},
allow programmers to skip defining the types of variables,
by doing type inferencing at compile time.
Others, like Python, also allow the programmer to avoid defining
the type of a variable, but these types are dynamic\cite{pythonsummary}, 
and therefore not the topic of the currrent report.

These languages use the Hindley-Milner type system, which has two main implementations, Algorithm W and Algorithm M.  (cite something)  Algorithm W is standard algorithm, first done by ( somebody ), while Algorithm M was not formally presented until 1998 by Oukseh Lee and Kwangkeun Yi\cite{Lee:1998:PFL:291891.291892}.

% explain the difference between M and W



% let-polymorphic is something that you might come across in reading
% it's only really relevent when polymorphism is available 
% (e.g.) f :: a -> a, where a could be any type
% this is not the case in our toy language, so don't worry about it for now
% if there's time later, we'll get to this



% swift type inferencing description; maybe useful as a reference??? https://github.com/apple/swift/blob/master/docs/TypeChecker.rst


\bibliographystyle{ieeetr}
\bibliography{bib.bib}

\end{document}
